\documentclass[10pt,a4paper]{article}
\usepackage[utf8]{inputenc}

\title{%
  Rate of Convergence: Fictitious Play \\
  \large Max Rapp, Silvan Hungerbuehler}

\usepackage{mathptmx} % "times new roman"
\usepackage{amssymb}
\usepackage{amsmath, amsthm}
\usepackage{amsfonts}
\usepackage{enumitem}
\usepackage{verbatim}
\usepackage{hyperref}
\usepackage{comment}
\usepackage[margin=1in]{geometry}
\usepackage[normalem]{ulem}
\date{}
\begin{document}
\maketitle

\subsection*{Introduction}
We have generated a large number of two-player, zero-sum games and simulated fictitious play on them to analyse their rate of convergence. 
In this report we briefly discusses the method used to generate different types of games and then show the results of our analysis concerning the rate of convergence. Specifically, we have found a systematic relationship between a game having pure Nash Equilibria, its rate of convergence and the number of iterations the fictitious play algorithm requires to terminate (given a certain convergence treshold).
\subsection*{Generating Games}
We used three types of two-player, zero-sum games for our analysis. The first type was constructed by randomly picking four integers  $\alpha,\beta,\gamma,\delta$ in the interval $[-1000,1000]$ to represent the row players' payoffs (ordered clockwise, starting in the top-left of the normal form). Since we only consider zero-sum games, we simultaneously determined the column player's utilities as $-\alpha,-\beta,-\gamma,-\delta$.
\begin{table}[h]
\centering
\caption{}
\begin{tabular}{|l|l|l|}
\hline
  & L                & R                \\ \hline
T & $\alpha,-\alpha$ & $\beta,-\beta$   \\ \hline
B & $\gamma,-\gamma$ & $\delta,-\delta$ \\ \hline
\end{tabular}
\end{table}

The second type of games satisfies the constraint that there are no strictly dominant strategies for either player. We generated this type by randomly picking four integers $\alpha,\beta,\gamma,\delta \in [-1000,1000]$ again and discarding all game where the constraint is violated. The third type of games is generated the same way, but this time under the constraint that there be no Nash Equilibria in dominant strategies. That is, maximally one player was allowed to have a dominant pure strategy.\\
We worked with these three types of games to assess if and how they would differ in terms of rate of convergence.
\subsection*{Analysis}
Since the actual mixed Nash Equilibria of the games were unknown to us, we used the following formula from Wikipedia to approximate the rate of convergence. $\boldsymbol{s}_l$ denotes the sequence of empirical-mixed-strategy profiles during fictitious play.
\begin{equation*}
p\approx\frac{\log{|\frac{s_{n+1}-s_{n}}{s_{n}-s_{n-1}}|}}{\log{|\frac{s_{n}-s_{n-1}}{s_{n-1}-s_{n-2}}|}}
\end{equation*}

\subsection*{Findings}
We find that games have a rate of convergence that is infinite \textbf{iff} they have a pure NE \textbf{iff} the fictitious play algorithm needs either 2 or $\tfrac{1}{\epsilon*10}$ iterations to terminate.\\
A game has no Nash equilibrium in pure strategies \textbf{iff} the rate of convergence equals $1$. In the absence of pure NE, the number of iterations seems to be upper bounded by roughly 5000 (for $\epsilon=0.0001$).

\textbf{References}\\
\url{https://en.wikipedia.org/wiki/Rate_of_convergence}

\end{document}